\chapter*{TÓM TẮT}
\addcontentsline{toc}{chapter}{TÓM TẮT}
Việc xây dựng hệ thống có khả năng tự động chơi game đang gây được sự chú ý lớn trong cộng động khoa học công nghệ, đặc biệt trong lĩnh vực Trí tuệ nhân tạo. 
Thực vậy, việc học để hiểu được các hình ảnh đầu vào trừu tượng của game để có thể chơi cho tốt là vấn đề tồn tại đã lâu trong lĩnh vực này.
Ngoài ra, việc xây dựng một hệ thống có thể chơi game tốt (đạt điểm cao ngang hoặc hơn con người) cũng chính là một cột mốc quan trọng trong lĩnh vực Trí tuệ nhân tạo.

Trong những năm gần đây, học sâu đã tạo ra được nhiều đột phá trong nhiều lĩnh vực: thị giác máy tính, xử lý ngôn ngữ tự nhiên, nhận dạng giọng nói... 
Theo xu hướng nghiên cứu hiện nay, khóa luận tập trung tìm hiểu cách giải quyết bài toán tự động chơi game bằng phương pháp học tăng cường kết hợp với học sâu. 
Dựa trên cơ sở của bài báo khoa học ``Human-level control through deep reinforcement learning'' được đăng trên tạp trí Nature số 518, khóa luận tiếp cận theo hướng sử dụng mô hình ``end-to-end'' (chỉ sử dụng mô hình duy nhất, việc học đặc trưng là tự động và gắn liền với việc học cách chơi game) để giải quyết bài toán tự động chơi game.

Kết quả sơ bộ của khóa luận đạt được đó là cài đặt lại mô hình đã được đề xuất trong bài báo ``Human-level control through deep reinforcement learning'' và đạt được kết quả tương đương được báo cáo trong bài báo này ở một số game thử nghiệm. 
Đồng thời khóa luận cũng phân tích khả năng học của mô hình cũng như những hạn chế vốn có của nó.