\chapter*{TÓM TẮT}
\addcontentsline{toc}{chapter}{TÓM TẮT}
Việc xây dựng hệ thống có khả năng tự động chơi game đang gây được sự chú ý lớn trong cộng động khoa học công nghệ, đặc biệt trong lĩnh vực sản xuất game. Thực vậy những hệ thống có khả năng tự động chơi game có tiềm năng rất cao như giúp cho nhà sản xuất game có thể đánh giá độ khó trong game, giải quyết sớm những lỗi phát sinh trong quá trình chơi trước khi họ phát hành game đó ra thị trường. Để có một giải pháp tổng quá giúp cho hệ thống có khả năng tự động chơi game vẫn đang là một bài toán khó trong cộng đồng trí tuệ nhân tạo.

Trong những năm gần đây, học sâu đã tạo ra được nhiều đột phá trong nhiều lĩnh vực: thị giác máy tính, xử lý ngôn ngữ tự nhiên, nhận dạng giọng nói... Theo xu hướng nghiên cứu hiện nay, khóa luận tập trung tìm hiểu cách giải quyết bài toán tự động chơi game bằng phương pháp học tăng cường kết hợp với học sâu. Dựa trên cơ sở của bài báo khoa học ``Human-level control through deep reinforcement learning'' được đăng trên tạp trí Nature số 518, khóa luận tiếp cận theo hướng sử dụng mô hình ``end-to-end'' (chỉ sử dụng mô hình duy nhất không qua các bước rút trích đặc trưng phức tạp) để giải quyết cho bài toán tự động chơi game.

Kết quả sơ bộ của khóa luận đạt đươc đó là cài đặt lại thành công mô hình đã được đề xuất trong bài báo ``Human-level control through deep reinforcement learning''. Đồng thời khóa luận cũng phân tích khả năng học của mô hình, cũng như những hạn chế vốn có của nó.