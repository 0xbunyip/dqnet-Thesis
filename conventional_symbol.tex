\chapter*{KÝ HIỆU - QUY ƯỚC}
\addcontentsline{toc}{chapter}{KÝ HIỆU - QUY ƯỚC}
\begin{tabular}{ l l }
	$s, s'$ & Những trạng thái\\
	$a$ & Hành động\\
	$r$ & Điểm thưởng\\
	$\mathcal{S}$ & Tập các trạng thái\\
	$\mathcal{R}$ & Tập các điểm thưởng có thể nhận được\\
	$\mathcal{A}(s)$ & Tập các hành động có thể thực hiện tại trạng thái $s$\\
	$t$ & Một thời điểm xác định\\
	$\mathit{A}_{t}$ & Hành động thực hiện tại thời điểm $t$\\
	$\mathit{S}_{t}$ & Trạng thái tại thời điểm $t$\\
	$\mathit{R}_{t}$ & Điểm thưởng nhận được tại thời điểm $t$\\
	$\mathcal{G}_{t}$ & Tổng điểm thưởng nhận được từ sau thời điểm $t$\\
	$\pi$ & Chính sách thực thiện\\
	$\pi(s)$ & Hành động được thực hiện tại trạng thái $s$ theo chính sách đơn định\\
	$\pi(a|s)$ & Xác suất hành động $a$ được thực hiện tại trạng thái $s$\\
	$v_{\pi}(s)$ & Giá trị của trạng thái $s$ dưới chính sách $\pi$ \\
	$v_{*}(s)$ & Giá trị trạng thái $s$ dưới chính sách tối ưu \\
	$q_{\pi}(s,a)$ & Giá trị của hành động $a$ tại trạng thái $s$ dưới chính sách $\pi$\\
	$q_{*}(s,a)$ & Giá trị của hành động $a$ tại trạng thái $s$ dưới chính sách tối ưu\\
	$V, V_{t}$ & Ước lượng hàm giá trị trạng thái $v_{\pi}$ hoặc $v_{*}$\\
	$Q, Q_{t}$ & Ước lượng hàm giá trị hành động $q_{\pi}$ hoặc $q_{*}$\\
	$\theta, \theta_{t}$ & Vec-tơ trọng số của một xấp xỉ hàm giá trị.
\end{tabular}